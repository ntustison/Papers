%% This is file `elsarticle-template-1-num.tex',
%%
%% Copyright 2009 Elsevier Ltd
%%
%% This file is part of the 'Elsarticle Bundle'.
%% ---------------------------------------------
%%
%% It may be distributed under the conditions of the LaTeX Project Public
%% License, either version 1.2 of this license or (at your option) any
%% later version.  The latest version of this license is in
%%    http://www.latex-project.org/lppl.txt
%% and version 1.2 or later is part of all distributions of LaTeX
%% version 1999/12/01 or later.
%%
%% The list of all files belonging to the 'Elsarticle Bundle' is
%% given in the file `manifest.txt'.
%%
%% Template article for Elsevier's document class `elsarticle'
%% with numbered style bibliographic references
%%
%% $Id: elsarticle-template-1-num.tex 149 2009-10-08 05:01:15Z rishi $
%% $URL: http://lenova.river-valley.com/svn/elsbst/trunk/elsarticle-template-1-num.tex $
%%

%\documentclass[preprint,authoryear,review,12pt]{elsarticle}
\documentclass[final,5p,times,twocolumn]{elsarticle}

%% Use the option review to obtain double line spacing
%% \documentclass[preprint,review,12pt]{elsarticle}

%% Use the options 1p,two column; 3p; 3p,twocolumn; 5p; or 5p,twocolumn
%% for a journal layout:
%% \documentclass[final,1p,times]{elsarticle}
%% \documentclass[final,1p,times,twocolumn]{elsarticle}
%% \documentclass[final,3p,times]{elsarticle}
%% \documentclass[final,3p,times,twocolumn]{elsarticle}
%% \documentclass[final,5p,times]{elsarticle}
%% \documentclass[final,5p,times,twocolumn]{elsarticle}


\usepackage{color}
\usepackage{multirow,booktabs,ctable,array}
\usepackage{lscape}
\usepackage{amsmath}
\usepackage{lineno}
\usepackage{ulem}
\usepackage{setspace}
\usepackage{listings}
\usepackage{float}
\usepackage{listings}
\usepackage{color,colortbl}
\usepackage{rccol}
\usepackage[table]{xcolor}

    \definecolor{listcomment}{rgb}{0.0,0.5,0.0}
    \definecolor{listkeyword}{rgb}{0.0,0.0,0.5}
    \definecolor{listnumbers}{gray}{0.65}
    \definecolor{listlightgray}{gray}{0.955}
    \definecolor{listwhite}{gray}{1.0}
    \definecolor{lightcyan}{rgb}{0.88,1,1}

\newcommand{\lstsetcpplong}
{
\lstset{frame = tb,
        framerule = 0.25pt,
        float,
        fontadjust,
        backgroundcolor={\color{listlightgray}},
        basicstyle = {\ttfamily\scriptsize},
        keywordstyle = {\ttfamily\color{listkeyword}\textbf},
        identifierstyle = {\ttfamily},
        commentstyle = {\ttfamily\color{listcomment}\textit},
        stringstyle = {\ttfamily},
        showstringspaces = false,
        showtabs = false,
        numbers = none,
        numbersep = 6pt,
        numberstyle={\ttfamily\color{listnumbers}},
        tabsize = 2,
        language=,
        floatplacement=!h,
        caption={\small \baselineskip 12pt DiReCT long command line menu which is invoked using the `{\ttfamily {-}{-}help}' option.  The short command line menu is obtained by typing `{\ttfamily {-}h}'},
        captionpos=b,
        label=listing:long
        }
}

\floatstyle{plain}
\newfloat{command}{thp}{lop}
\floatname{command}{Command}

%\usepackage[nomarkers,notablist]{endfloat}

%% if you use PostScript figures in your article
%% use the graphics package for simple commands
%% \usepackage{graphics}
%% or use the graphicx package for more complicated commands
%% \usepackage{graphicx}
%% or use the epsfig package if you prefer to use the old commands
%% \usepackage{epsfig}

%% The amssymb package provides various useful mathematical symbols
\usepackage{amssymb}
%% The amsthm package provides extended theorem environments
% \usepackage{amsthm}
 
 \usepackage{makecell}

%% The lineno packages adds line numbers. Start line numbering with
%% \begin{linenumbers}, end it with \end{linenumbers}. Or switch it on
%% for the whole article with \linenumbers after \end{frontmatter}.
%% \usepackage{lineno}

%% natbib.sty is loaded by default. However, natbib options can be
%% provided with \biboptions{...} command. Following options are
%% valid:

%%   round  -  round parentheses are used (default)
%%   square -  square brackets are used   [option]
%%   curly  -  curly braces are used      {option}
%%   angle  -  angle brackets are used    <option>
%%   semicolon  -  multiple citations separated by semi-colon
%%   colon  - same as semicolon, an earlier confusion
%%   comma  -  separated by comma
%%   numbers-  selects numerical citations
%%   super  -  numerical citations as superscripts
%%   sort   -  sorts multiple citations according to order in ref. list
%%   sort&compress   -  like sort, but also compresses numerical citations
%%   compress - compresses without sorting
%%
%% \biboptions{comma,round}

% \biboptions{}

\providecommand{\OO}[1]{\operatorname{O}\bigl(#1\bigr)}

\graphicspath{
             {./Figures/}
             }

\long\def\symbolfootnote[#1]#2{\begingroup%
\def\thefootnote{\fnsymbol{footnote}}\footnote[#1]{#2}\endgroup}



\journal{NeuroImage}

\begin{document}


\begin{frontmatter}

\title{Multivariate Neuroanalysis with ANTsR:  Application to Supervised Brain Segmentation with Concatenated Random Forests}

\author[label1]{Nicholas J.~Tustison
  \fnref{label0}}
  \fntext[label0]{\scriptsize Corresponding author:  PO Box 801339, Charlottesville, VA 22908; T:  434-924-7730; email address:  ntustison@virginia.edu.
  }
\author[label2]{K.~L.~Shrinidhi}
\author[label2]{Jeffrey T.~Duda}
\author[label2]{Philip A.~Cook}
\author[label1]{Christopher Durst}
\author[label1]{James C.~Gee}
\author[label1]{Murray C.~Grossman}
\author[label1]{Max Wintermark}
\author[label2]{Brian B.~Avants}
\address[label1]{Department of Radiology and Medical Imaging, University of Virginia, Charlottesville, VA}
\address[label2]{Penn Image Computing and Science Laboratory, University of Pennsylvania,
                Philadelphia, PA}

%\maketitle

\linenumbers

\begin{abstract} 
Improvements in image acquisition and increasingly sophisticated statistical techniques 
for neuroanalysis have spurred the development of advanced computational frameworks for studying the
brain---many of which
have been made publicly available.  One such popular toolkit is the open source Advanced
Normalization Tools (ANTs) which contains a suite of well-vetted processing tools for
core data transformation tasks such as image registration, image segmentation, 
inhomogeneity field correction and template building.  However, despite the numerous
solutions afforded by such tools, neuroscience (and data analysis in general) requires
a statistical platform for making inferences with respect to given hypotheses once the 
requisite data transformations have been performed.  In this paper, we describe the
coupling of ANTs with the widely-used R language---an open source environment for 
statistical processing and data visualization---which we denote as ANTsR.  One of the
many benefits from such an integration is the accessibility to advanced statistical 
and machine learning techniques.  To showcase the flexibility and power of ANTsR, we 
apply the combination of one such technique, Random Forests, and ANTs processing 
tools to the difficult problem of brain tumor segmentation.  This includes evaluation on 
the public data set from the MICCAI 2012 BRATS challenge consisting of both real and simulated
data.  To facilitate reproducibility, all scripts and data used in this paper are 
publicly available for download.  
\end{abstract}

\begin{keyword}
advanced normalization tools \sep BRATS \sep brain tumor segmentation \sep R project \sep random forests
%% keywords here, in the form: keyword \sep keyword
\end{keyword}

\end{frontmatter}
%
%
\newpage

%% MSC codes here, in the form: \MSC code \sep code
%% or \MSC[2008] code \sep code (2000 is the default)

%%
%% Start line numbering here if you want
%%
% \linenumbers

%% The Appendices part is started with the command \appendix;
%% appendix sections are then done as normal sections
%% \appendix

%% \section{}
%% \label{}

%% References
%%
%% Following citation commands can be used in the body text:
%% Usage of \cite is as follows:
%%   \citep{key}          ==>>  [#]
%%   \cite[chap. 2]{key} ==>>  [#, chap. 2]
%%   \citet{key}         ==>>  Author [#]

%% main text

%\input{intro}
%
%\section{Methods and Materials}

\subsection{ANTs volumetric-based cortical thickness estimation pipeline}

The ANTs-based cortical thickness estimation workflow is illustrated 
in Figure \ref{fig:pipeline}.  The steps are as follows:
\begin{enumerate}
  \item initial N4 bias correction on input anatomical MRI,
  \item brain extraction using a hybrid segmentation/template-based strategy,
  \item alternating between prior-based segmentation and white matter posterior
        probability weighted bias correction,
  \item DiReCT-based cortical thickness estimation, and
  \item optional normalization to specified template.
\end{enumerate}
Each component, including both software and data, is briefly detailed 
below with the relevant references for additional information. 


%We also note that each component is publicly available with all ANTs 
%algorithms available as open source.%
%\footnote{
%http://www.picsl.upenn.edu/ANTS
%}
Additionally, the coordination of all the algorithmic components is
encapsulated in the shell script \verb#antsCorticalThickness.sh#.  This includes
optimal parameters for each of the algorithmic components which has worked
well for our processing and which are used to acquire the results 
described in this work.

%\lstset{frame = tb,
%        framerule = 0.25pt,
%        float,
%        fontadjust,
%        backgroundcolor={\color{listlightgray}},
%        basicstyle = {\ttfamily\scriptsize},
%        keywordstyle = {\ttfamily\color{listkeyword}\textbf},
%        identifierstyle = {\ttfamily},
%        commentstyle = {\ttfamily\color{listcomment}\textit},
%        stringstyle = {\ttfamily},
%        showstringspaces = false,
%        showtabs = false,
%        numbers = none,
%        numbersep = 6pt,
%        numberstyle={\ttfamily\color{listnumbers}},
%        tabsize = 2,
%        language=,
%        floatplacement=!h,
%        caption={\small \baselineskip 12pt DiReCT long command line menu which is invoked using the `{\ttfamily {-}{-}help}' option.  The short command line menu is obtained by typing `{\ttfamily {-}h}'}.,
%        captionpos=b,
%        label=listing:long
%        }
%\lstsetcpplong
%\begin{lstlisting}
%This script, apb.sh, performs T1 anatomical brain 
%processing where the following steps are currently 
%applied:
%
%  1. Brain extraction
%  2. Brain 3-tissue segmentation
%  3. Cortical thickness
%  4. (Optional) registration to a template
%
%Usage:
%
%abp.sh -d ImageDimension
%       -i T1Image.nii.gz
%       -e BrainExtractionTemplate
%       -m BrainExtractionProbabilityMask
%       -l BrainParcellationTemplate
%       -p BrainParcellationProbabilityMask
%       <OPTARGS>
%       -o OutputPrefix
%
%Example:
%
%abp.sh -d 3 
%       -i t1.nii.gz 
%       -e brainWithSkullTemplate.nii.gz 
%       -m brainPrior.nii.gz 
%       -l corticalLabels.nii.gz 
%       -p corticalLabelPriors.nii.gz 
%       -o output
%
%Compulsory arguments:
%
%     -d:  ImageDimension                        2 or 3 (for 2 or 3 dimensional single image)
%     -a:  Anatomical T1 image                   typically T1.
%     -e:  Brain extraction template             Anatomical template created using e.g. LPBA40 data set with
%                                                buildtemplateparallel.sh in ANTs.
%     -m:  Brain extraction probability mask     Brain probability mask created using e.g. LPBA40 data set which
%                                                have brain masks defined, and warped to anatomical template and
%                                                averaged resulting in a probability image.
%     -l   Brain segmentation template           Anatomical template for brain segmentation.  E.g. NIREP template
%                                                with labels.
%     -p   Brain segmentationpriors              Label probability priors corresponding to the image specified
%                                                with the -l option.  Specified using c-style formatting, e.g.
%                                                -p labelsPriors\%02d.nii.gz.
%     -o:  OutputPrefix                          The following images are created using the specified prefix:
%                                                  * /Users/ntustison/Data//tmp13243//tmpN4Corrected.nii.gz
%                                                  * /Users/ntustison/Data//tmp13243//tmpExtractedBrain.nii.gz
%                                                  * /Users/ntustison/Data//tmp13243//tmp3TissueBrainSegmentation.nii.gz
%                                                  * /Users/ntustison/Data//tmp13243//tmpCorticalThickness.nii.gz
%                                                  * /Users/ntustison/Data//tmp13243//tmpSurfaceCurvature.nii.gz
%
%Optional arguments:
%
%     -s:  image file suffix                     Any of the standard ITK IO formats e.g. nrrd, nii.gz (default), mhd
%     -t:  template for t1 registration
%     -k:  keep temporary files                  Keep brain extraction/segmentation warps, etc (default = false).
%     -w:  white matter label                    white matter label for segmentation (default = 3).
%     -g:  gray matter label                     cortical gray matter label for segmentation (default = 2)
%     -i:  max iterations for registration       ANTS registration max iterations (default = 50x100x20)
%\end{lstlisting}



\begin{figure*}
  \centering
  \includegraphics[width=130mm]{Figures/Kapowski_pipeline.pdf}
  \caption{The ANTs T1 processing workflow containing all elements for 
  determining cortical thickness. Not shown is the optional single subject
  to template registration.}
  \label{fig:pipeline}
\end{figure*}

\subsubsection{Anatomical template construction}

Normalizing images to a standard coordinate system
reduces intersubject variability in population studies.  Various
approaches exist for determining the normalized space such as the selection
of a pre-existing template based on a single subject, e.g. the Talairach
atlas \citep{Talairach1988}, or a publicly available averaged group of
subjects, e.g. the MNI \citep{Collins1994} or ICBM \citep{Mazziotta1995}
templates.  Additionally, mean templates constructed from labeled
data can be used to construct spatial priors for improving segmentation
algorithms.
The work of \cite{avants2010} explicitly models the geometric component of the 
normalized space during optimization to produce such mean templates.  Coupling the intrinsic symmetry of 
SyN pairwise registration \citep{avants2011} and an
optimized shape-based sharpening/averaging of the template appearance, Symmetric Group
Normalization (SyGN) is a powerful framework for producing optimal population-specific
templates \citep{avants2010} with arbitrary similarity metric choice.  

%One challenge with standard templates is that they may inadvertently bias one's results by enabling better normalization of subjects to which the template is more similar.  This issue is exacerbated when dealing with populations that have high variance (e.g. due to disease) and/or when one's normalization method is low-dimensional (not flexible enough to capture large shape differences). 

%Population-specific templates alleviate some of
%the issues with other template approaches by deriving a most representative image from the population
%\citep{Good2001}.  Large deformation registration algorithms also reduce
%this confound by being less sensitive to the deformation distance
%between subject and target.  Some approaches combine both advantages,
%for instance, the diffeomorphic approach of Joshi et al. employs the
%SSD metric and a shape distance to bring the subject group of images
%into alignment \citep{Joshi2004}.  Variants
%include extension to multiple modalities \citep{Lorenzen2006} and small deformations
%\citep{Geng2009}.  These approaches iteratively minimize group difference in ``congealing''
%towards a representative image template \citep{Learned-Miller2006}.


The ANTs implementation of this technique is currently available as a shell script, 
\verb#buildtemplateparallel.sh#, and a multivariate version,
\verb#antsMultivariateTemplateConstruction.sh#, both of which are distributed as part of
 the ANTs repository.
The latter script permits the construction of multimodal templates (e.g. 
T1-weighted, T2-weighted, and proton density MRI as described in the 
Evaluation section).  Both scripts accommodate a variety of computational resources
for facilitating template construction.  These computational resource possibilities include:
\begin{itemize}
  \item serial processing on a single workstation, 
  \item parallelized processing on a single workstation with multiple cores using \verb#pexec#%
  \footnote{http://www.gnu.org/software/pexec/pexec.1.html},
  \item parallelized processing using Apple's XGrid technology%
  \footnote{https://developer.apple.com/hardwaredrivers/hpc/xgrid\_intro.html}, 
  \item parallelized processing using Sun Grid Engine for cluster-based systems%
  \footnote{http://www.oracle.com/technetwork/oem/grid-engine-166852.html}, and 
  \item parallelized processing using the Portable Batch System for cluster-based systems%
  \footnote{http://www.pbsworks.com/}.
\end{itemize}
Within this work multiple templates were created for all stages of 
image processing and analysis.  The creation of these templates are described
in the corresponding data section.

\textcolor{blue}{what is the motivation of the multivar template here?
  it would seem useful to add an FA component to the segmentation step
  to aid in cortical / wm delineation which would motivate the
  multivar template approach but might be difficult to establish that
  this actually helps .... though we could compare both ways : univar
  and multivar }

%Given a set of representative images, 
%$\{I_1, I_2, \ldots, I_M\}$, optimization involves finding the set of paired
%diffeomorphic transformations, $\left\{\left(\phi^1_1, \phi^1_2\right), 
%\left(\phi^2_1, \phi^2_2\right), \ldots, \left(\phi^M_1, \phi^M_2\right) \right\}$,
%the optimal template appearance, $J^*$, and corresponding coordinate system, $\psi(\mathbf{x})$,
%which minimize the following cost function:
%\begin{align}
%  \sum_{m=1}^M \left[
%    D\left( \psi(\mathbf{x}), \phi^m_1(\mathbf{x}, 1) \right) + 
%    \Pi \left( I_m\left(\phi^m_2(\mathbf{x}, 0.5) \right), J^*\left(\phi^m_1(\mathbf{x}, 0.5) \right) \right)
%    \right]
%\end{align}
%where $D$ is the diffeomorphic shape distance, 
%\begin{align}
%  D\left(\phi(\mathbf{x}, 0), \phi(\mathbf{x}, 1)\right) = \int_{0}^1 \| v(\mathbf{x}, t) \|_L dt, 
%\end{align}
%dependent upon the choice of the linear operator, $L$, and
%$v$ is the diffeomorphism-generating velocity field, 
%\begin{align}
%  v\left(\phi(\mathbf{x}, t), t \right) = \frac{d\phi(\mathbf{x}, t)}{dt},\,\,\, \phi(\mathbf{x}, 0) = \mathbf{x}.
%\end{align}
%$\Pi$ is th
%e choice of similarity metric, often cross-correlation \citep{Avants2008a}, calculated in the 
%virtual domain midway between each individual image and the current estimate of the template. 
%
%With initial assignment of $\left\{\left(\phi^m_1, \phi^m_2\right)\right\}$ and $\psi(\mathbf{x})$ 
%to identity, iterative optimization
%involves estimating the pairwise transformations, estimation of the optimal template appearance, and 
%updating $\psi(\mathbf{x})$ by averaging the current estimate of $\left\{\phi_1^m\right\}$.  


\subsubsection{N4 Bias field correction}

Critical to quantitative processing of MRI is the minimization of
field inhomogeneity effects which causes artificial low frequency 
intensity variation across the image.  Large-scale studies, such
as the Alzheimer's Disease Neuroimaging Initiative (ADNI), employ
perhaps the most widely used bias correction algorithm, N3 \cite{sled1998}, 
as part of their standard protocol \citep{boyes2008}.

In \cite{tustison2010}, we introduced an extension of N3, denoted as
N4, which demonstrates improved performance and convergence behavior
on a variety of data.  This improvement is a result of an enhanced 
fitting routine (which includes multi-resolution capabilities) and a modified optimization 
formulation.  For our workflow, the additional possibility of specifying
a weighted mask in N4 permits the use of the current white matter probability map 
calculated during the segmentation pipeline for further improvement of 
bias field estimation.  In addition to its public availability 
through ANTs and the Insight Toolkit, it has also been included in
the popular open source Slicer software package for visualization and medical
image computing \cite{fedorov2011}.

N4 is used in two places during the individual subject processing (cf. Figure
\ref{fig:pipeline}).  Following conversion of the raw dicom T1-weighted image
to Nifti format using our related \verb#Neuropipedream# set of raw image conversion
and organization tools%
\footnote{
http://sourceforge.net/projects/neuropipedream/
}, N4 is used to generate an initial bias corrected image for use in
brain extraction.  The input mask is created by adaptively thresholding 
the background from the foreground using Otsu's algorithm \cite{otsu1979}.
Following brain extraction, the three-tissue segmentation involves iterating
between bias field correction using the current white matter posterior 
probability as a weight mask and then using that bias corrected image
as input to the Atropos segmentation step (described in subsequent sections). 

\subsubsection{Atropos 3-tissue segmentation}

In \cite{avants2011a} we presented an open source $n$-tissue segmentation software tool
(which we denote as ``Atropos'') attempting to distill 20+ years of active research in this area
particularly some of its most seminal work (e.g. \cite{zhang2001,ashburner2005}). 
Specification of prior probabilities includes spatially varying Markov Random Field modeling, 
prior label maps, and prior probability maps typically derived from our template building 
process.  Additional
capabilities include handling of multivariate data, 
partial volume modeling \cite{shattuck2001}, a memory-minimization mode,
label propagation, a plug-n-play architecture for incorporation of novel likelihood models
which includes both parametric and non-parametric models for both scalar and tensorial
images, and alternative posterior formulations for different segmentation tasks.

\subsubsection{Brain extraction}

Brain extraction using ANTs combines template building, high-performance
brain image registration \citep{avants2011}, and Atropos with topological refinements.  
An optimal template for brain extraction is 
generated offline using labeled brain data.  For example, in this work we use the LPBA40 data 
for generating a brain extraction template and a corresponding brain probability mask which is
available on the website associated with this submission. 

  The warped template probability map is thresholded at 0.5 and the resulting mask is dilated
with a radius of 2.  Atropos is used to generate an initial 3-tissue segmentation estimate within the mask
region.  Each of the three tissue masks undergo specific morphological operations which are then
combined to create a brain extraction mask for use in the rest of the
cortical thickness workflow.  \textcolor{blue}{does there need to be a
  bit more technical detail here?  while you can refer to the script,
  why perform these operations?  there are numerous references, most
  germane probably some recent stuff from J Prince ( i think ) and the
  freesurfer watershed approach which came from ... i forget ... maybe
  one of the french groups.}

A comparison using open access brain data with publicly available
brain extraction algorithms including AFNI's \verb#3dIntracranial#
\citep{ward1999}, FSL's \verb#BET2# \citep{smith2002}, Freesurfer's
\verb#mri_watershed# \citep{segonne2004}, and BrainSuite
\citep{dogdas2005} demonstrated that our combined
registration/segmentation approach \citep{avants2010a} performs at the
top level alongside BrainSuite (tuned) and FreeSurfer.
\textcolor{blue}{ok you have the segonne ref here ... }


\subsubsection{DiReCT Cortical Thickness Estimation}

Although the basic formulation of DiReCT as reported in this work is as it 
was introduced in \cite{das2009}, we have made several 
improvements.  Perhaps the most significant advance is that this particular 
ITK-compatible implementation has been significantly multi-threaded,
is written in ITK coding style, and has been made publicly available through 
ANTs complete with a unique user interface design developed specifically for 
ANTs tools.  

\subsection{Public Data Resources}

\subsubsection{LPBA40 Data for Skull Stripping}

For the brain extraction step
we used the data from the LPBA40 repository \citep{shattuck2008}.
These data consist of 40 high-resolution 3D Spoiled Gradient Echo
(SPGR) MRI acquisitions which were manually labeled delineating
56 brain structures.  Additional post processing 
included automated brain extraction using FSL's brain extraction tool 
\citep{smith2002} which was followed by manual corrections.  These
40 brain masks are also included with the database.  

All 40 subjects were used to create a population-specific
unbiased average template \cite{avants2010}.  The brain masks corresponding 
to the 20 subjects were warped to the template space using the 
transforms derived during the template building process.  A template 
probability mask was created by averaging the warped brain masks.
For brain extraction of any single individual this ANTS-based LPBA40 template 
is coarsely registered to the single subject brain.  The template
probability brain mask is warped to the individual subject and 
is used as the initial brain mask estimate.  As mentioned previously,
Atropos and binary morphological operations are used to refine the
brain mask estimate.

\subsubsection{NIREP Data for 3-Tissue Segmentation}

The nonrigid image registration evaluation project is an ongoing 
framework for evaluating image registration algorithms \citep{christensen2006}.
The initial data set introduced into the project consists of 
16 (8 male and 8 female) high resolution skull-stripped brain 
data with 32 cortical labels manually drawn using published protocol.
Given the gray matter labels, the white matter and CSF were identified 
for each of the 16 subjects using Atropos.  Similar to the LPBA40
data set, a NIREP template was created from all 16 subjects and each the
warped labels were used to create probabilistic estimates of the 
labeled region boundaries. These probability maps were used as 
spatial prior probabilities during the 3-tissue segmentation component
of the pipeline.  Using SyN, the NIREP template is registered to the
extracted individual subject brain which is followed by a warping of the 
NIREP priors to the space of the individual subject.  The initial warped 
white matter probability map is used as the weighted confidence mask 
in the follow-up bias correction step.

\subsection{IXI Data for Pipeline Evaluation}

The IXI data%
\footnote{
http://biomedic.doc.ic.ac.uk/brain-development/
}
consists of approximately 600 healthy subjects imaged at three sites 
using several modalities (T1-weighted, T2-weighted, proton density, magnetic 
resonance angiography, and diffusion tensor imaging).  The 
database also consists of  demographic information such as age, weight,
height, ethnicity, occupation category, educational level, and marital status.
The number of subjects spanning a range of demographic characteristics makes
this a rich data set for validating and exploring correlations with cortical 
thickness measured using the ANTs pipeline.

\begin{figure}
  \centering
  \includegraphics[width=90mm]{Figures/template.pdf}
  \caption{Sample multivariate template constructed from a subset of the IXI data (female, age 40--50).  Axial slices of five of the 37 total subjects from this cohort are shown. }
  \label{fig:template}
\end{figure}



\section{Introduction}

ANTs (Advanced Normalization Tools) originated with the open source availing
of state-of-the-art registration algorithms for neuroimage analysis
\cite{avants2008a} built upon the mature and well-vetted Insight Toolkit
of the National Institutes of Health.  Since then, the toolkit has grown to include 
several algorithmic solutions necessary
for robust medical image analysis including bias correction \cite{tustison2010}, 
$n$-tissue multivariate segmentation \cite{avants2011}, template construction \cite{avants2010}, 
and cortical thickness estimation \cite{das2009} (many of which have been
introduced into ITK partially in an attempted leveraging of Linus's Law).%
\footnote{
``Given enough eyeballs, all bugs are shallow.'' --Linus Torvalds
}   
However, in the evolution of the toolkit, it became clear (as neuroimaging
research certainly falls under the popular rubric of ``big data analysis'')
that robust statistical machinery was lacking for proper inferences regarding
the data produced by the various ANTs tools.
ANTsR was developed
specifically to provide an interface between a 
powerful neuroimaging toolkit for producing reliable imaging data 
transformations and the R project%
\footnote{
http://www.r-project.org
}
for statistical computing and visualization thus providing a complete
set of tools for multivariate image analysis. 

In addition to describing ANTsR basics and how it can be generally
used in multivariate neuroimaging studies, we showcase its use in
a particularly salient application for performing supervised brain
segmentation using random forests.  Although random forests have
been proposed previously in the literature for supervised brain 
segmentation (e.g. \cite{geremia2011,zikic2012}), we demonstrate that ANTsR
significantly facilitates the construction of a fully functional, 
parallelizable workflow for such a task and that performance 
exceeds that of the top competitors in the recent Multimodal Brain
Tumor Segmentation Challenge as part of the MICCAI 2012 conference.%
\footnote{
http://www2.imm.dtu.dk/projects/BRATS2012/  
}

\section{Materials and Methods}

\subsection{ANTsR:  An ANTs/R Interface}

\subsubsection{Installation}

The ANTsR package is publicly available on the github project hosting service.%
\footnote{
https://github.com/stnava/ANTsR
}
Prior to installation of ANTsR, several external R packages
need to be installed including: \verb#Rcpp#, \verb#signal#, \verb#timeSeries#, 
\verb#mFilter#, \verb#doParallel#, \verb#robust#, \verb#magic#, \verb#knitr#, \verb#pixmap#, 
\verb#rgl#, \verb#misc3d# which is facilitated by the 
\verb#install.packages()# mechanism.  Additionally, in order
to perform the supervised brain segmentation as described 
in later sections, one would need to also install 
\verb#randomForest#, \verb#snowfall#, \verb#rlecuyer#,
and \verb#ggplot2#. 

CMake%
\footnote{
http://www.cmake.org/
}
is an open source tool for the management and building of 
large-scale software projects.  It is used
to coordinate the downloading of external packages,
such as the Insight Toolkit (ITK)%
\footnote{
http://www.itk.org/
}
and ANTs.  Detailed instructions for download and
installation can be found on the ANTsR github website.

\subsection{Supervised Brain Segmentation}

Given a set of training data consisting of 
labeled (csf, gray matter, white matter, edema, and tumor),
multimodal brain image data, 
supervised brain segmentation entails the generation of a 
statistical model from such training data which can
then be extended to testing data, i.e. unlabeled 
brain data.  In subsequent sections, we introduce the generic
{\it random forest} modeling framework which takes as input
a set of {\it feature images} (also described in a later section)
and labels for each training subject 
and outputs a statistical model.  This model, in turn, can then
be used to provide a probabilistic estimate of the labels in an
unlabeled subject. 

One of the core extensions that we provide in this work is
the use of concatenated random forests for improved probabilistic estimation
of the labels.
As we demonstrate in the Results section, the set of feature
images employed work sufficiently well for good performance
on the training data (which exceeds that of what has been
currently published in the literature).  However, we discovered
that further improvements could be gained by using the probabilistic 
label estimates as input to enhance preprocessing before a 
second round of feature image generation including 
modality-specific, prior-based $n$-tissue segmentation and 
random forest model creation. 

\subsubsection{Random Forests}

Several machine learning concepts were integrated to create 
the random forests framework first articulated in its entirety by Breiman
et al. \cite{breiman2001} for performing classification/regression.  
Although decision trees had been previously explored in the literature, 
it was the success of ``boosting''-style machine learning 
techniques, such as AdaBoost \cite{schapire1990,freund1997}, which influenced 
the aggregation of such decision trees into ``forests'' 
with randomized node optimization for improved
classification/regression performance \cite{ho1995,amit1997}.
The final element of bootstrap aggregating or ``bagging'' (i.e.
random sampling of the training data) was
introduced by Breiman \cite{breiman1996} to achieve improved
accuracy.  

Early adoption \cite{viola2005} and success in the
computer vision community
has led to a recent surge within the medical image analysis
community of using random forests for handling complex 
classification/regression tasks including
normal brain segmentation \cite{yi2009},
MS lesion segmentation \cite{geremia2011}, 
multimodal brain tumor segmentation
\cite{zikic2012,geremia2012}, brain extraction \cite{iglesias2010}, 
anatomy detection in computed tomography \cite{criminisi2013}, and
segmentation of echocardiographic images \cite{verhoek2011}.
A thorough introduction for those interested in delving deeper 
into the more theoretical aspects of random forests can be found
in \cite{criminisi2011}.

One of the principal advantages of R is the extensive community of
developers  who have contributed on the order of thousands of packages 
extending R's capabilities beyond its core functionality.
Most relevant for the work described in this paper
is the \verb#randomForest# package developed from Breiman's original
Fortran code by Liaw and Wiener (described briefly in \cite{liaw2002}).


\subsubsection{Multi-Modal Feature Image Preprocessing}

Although several studies have pointed out the importance of
intensity normalization and bias correction, our experience 
with the training data illustrated a degradation in performance
when one or both steps (using \cite{nyul2000} and N4 \cite{tustison2010},
respectively) were performed due to the presence of the tumor/edema complex. 
 
As a corrective, for the first stage we simply windowed the image intensity
for all images to be between the quantiles $[0.01,0.99]$ and
subsequently rescaled to $[0,1]$.  From these ``corrected'' images,
the first set of feature images were derived.  For the training cohort, these
data were used to create the random forest regression model for the first
stage.  During the second stage, the probabilistic estimates
of the white matter and gray matter labels were used to generate a
``pure tissue weight mask'' to estimate the bias field 
using N4 (although the resulting bias field estimation was used
to correct the image within the entire cerebral mask).  Formally, this 
involved generation of a probabilistic map defined as:
\begin{align}
  P_{pure\,\,tissue}(\mathbf{x}) = \sum_{i=1}^N P_i(\mathbf{x}) \prod_{j=1, j \neq i}^N \left( 1 - P_j(\mathbf{x}) \right)
\end{align}
where $N$ is the set of user-selected tissue labels (in our
case $N=2$ consisting of the gray and white matter probability
maps).

Both rescaling and weighted bias correction were applied to produce
the ``corrected images'' for the second stage resulting in
modified features images for the second stage.  Note that we
perform a similar iterative scenario for normal brain 
segmentation \cite{avants2011} (encapsulated in the ANTs script 
\verb#antsAtroposN4.sh#).

\subsubsection{Multi-Modal Feature Image Generation}

Key to any supervised regression or classification protocol are the 
selected features for training and subsequent testing.  Based on previous
work and our own experience, we selected the following feature images
to showcase the supervised segmentation strategy developed in this work.

\begin{figure*}
  \centering
  \includegraphics[width=180mm]{Figures/featuresImages.pdf}
  \caption{Feature images from the BRATS\_HG0004 data set.}
\end{figure*}

\begin{itemize}
  \item Per modality (FLAIR, T1, T1C, T2)
    \begin{itemize}
      \item First-order neighborhood statistical images:
            mean, variance, skewness, and entropy. 
            Neighborhood radius = 3.
    \item GMM-based posteriors: CSF, gray matter, white matter, edema, and tumor
    \item GMM connected component geometry features:  volume, volume to 
          surface area ratio, eccentricity, elongation
    \item Template-based:  symmetric template difference and contralateral difference.
          Gaussian smoothing ($\sigma = 4$mm).
    \end{itemize}
  \item Not modality specific
    \begin{itemize}
    \item Normalized Euclidean distance
    \item Log Jacobian  
    \item T1C,T1 difference image
    \end{itemize}
\end{itemize}

For each modality, we create four first-order statistical feature images,
five Gaussian mixture model (GMM)-based posterior probability feature images,
four geometry features generated from the GMM posterior probability images
based on connected components, and two difference images using symmetric template
construction for a total of 4 modalities $\times$ (4 + 5 + 4 + 2) feature images per modality $=$ 60 total
feature images.  We employ two additional images consisting of the 
Euclidean distance image \cite{maurer2003} created from the skull-stripped 
binary mask rescaled to the range $[0,1]$ and the log
Jacobian image derived from the spatial normalization of the symmetric multivariate template and individual subject images.  Given the intensity corrected images
the corresponding multivariate template images, and a brain mask for each subject
creation of all feature images is performed using the script \verb#createFeatureImages.sh#.

Prior  cluster centers for specific tissue types learned from training data \cite{reynolds2009} are used in the GMM to create multiple feature images.  
Given $M$ tissue types (e.g. CSF, gray matter,
white matter, edema, and tumor), a GMM formulates the 
probability distribution at each voxel, $\mathbf{x}$, as the
sum of Gaussian components, $\mathcal{N}(\mathbf{x}|\mu,\sigma)$, i.e.
\begin{align}
p\left(\mathbf{x}|\mu_m,\sigma_m,\lambda_m\right) = \sum_{i=1}^M \lambda_m \mathcal{N}(\mathbf{x}|\mu_m,\sigma_m)
\end{align}
where $\sum_{m=1}^M \lambda_m = 1$.  One popular method for 
determining the parameters of the GMM is maximum likelihood 
estimation which can performed using the Atropos segmentation 
tool \cite{avants2011}.  In contrast to previous generative
modeling approaches for multi-modal tumor segmentation 
(e.g. \cite{prastawa2003,zikic2012}), we do not use multivariate 
Gaussians to specify tissue probabilities but rather incorporate each
univariate probability map into the feature vector of the training
data.  As pointed out in \cite{menze2010}, multivariate modeling
might obscure the distinct biological information provided by each 
modality.  Instead, we let the random forest construction 
process determine the optimal combination of such multivariate
information.
Additionally, maximum posterior labeling from the GMM processing
is used to determine the connected components for each label.  
Geometric features (assigned voxel-wise) include the physical volumes 
of each connected component including the volume to surface area ratio, 
the elongation, and eccentricity. 

In order to better characterize deviations from normal
multi-modal brain shape and appearance, several features were derived 
using population-specific multivariate template 
construction. A recent neuroimaging reproducibility study
by Landman et al. resulted in an open data cohort of 21
normal individuals, each imaged twice, comprising several
modalities including arterial spin labeling, 
fluid attenuated inversion recovery (FLAIR),
diffusion tensor imaging, functional imaging, T1, and T2 
\cite{landman2011}.

Given $K$ image modality types for $N$ subjects,  
${\mathbf I} = \{I_1,I_2,\ldots, I_K\}$, multivariate 
template construction iterates between optimizing the set 
of diffeomorphic transforms between the subjects and the 
template, 
$\left\{\left(\phi_1,\phi_1^{-1}\right),\ldots,\left(\phi_N,\phi_N^{-1}\right)\right\}$ 
and constructing the 
optimal multivariate template appearance 
$\mathbf{J}=\{J_1,J_2,\ldots, J_K\}$ to minimize the
following cost function:
\begin{align}
  \sum_{n=1}^N 
        \Bigg[ D &\left( \psi(\mathbf{x}),\phi_1^n(\mathbf{x},1)\right) \\ \nonumber 
        +& \sum_{k=1}^K \lambda_k \Pi_k \left(I_k^n\left(\phi_n(\mathbf{x},0.5)\right),J_k\left(\phi^{-1}_n(\mathbf{x},0.5)\right)\right)\Bigg]
\end{align}
where $D$ is the diffeomorphic shape distance,
\begin{align}
D\left( \phi( \mathbf{x},0),\phi( \mathbf{x},1)\right) = \int_0^1 \| \nu(\mathbf{x},t)\|_L dt
\end{align}
dependent on the choice of linear operator, $L$, and $\nu$
is the velocity field
\begin{align}
\nu\left( \phi(\mathbf{x},t) \right) = \frac{d\phi(\mathbf{x},t)}{dt},\,\,\, \phi(\mathbf{x},0) = \mathbf{x}.
\end{align}
Each pairwise registration employing the similarity metric $\Pi_k$ can 
be assigned a relative weighting, $\lambda_k$, to weight a particular
modality's influence in the construction process.  Further theoretical
details can be found in \cite{avants2008,avants2010}.
In terms of implementation, this algorithm is 
encapsulated in the script \verb#antsMultivariateTemplateConstruction.sh#,
available in the ANTs repository, which permits parallel processing on
an individual workstation or on a large computational cluster.

\begin{figure}
  \centering
  \begin{tabular}{cc}
    \includegraphics[width=40mm]{Figures/S_templateFA_140.png} &
    \includegraphics[width=40mm]{Figures/S_templateFLAIR_140.png} \\
    (a) & (b) \\
    \includegraphics[width=40mm]{Figures/S_templateT1_140.png} &
    \includegraphics[width=40mm]{Figures/S_templateT2_140.png} \\
    (c) & (d) 
  \end{tabular}
  \caption{Multivariate symmetric template created from the Kirby 
           21 data described in \cite{landman2011}.  Shown are the
           (a) fractional anisotropy (FA), (b) FLAIR, (c) MPRAGE, 
           and (d) T2 template components.
          }
  \label{fig:symmetrictemplates}
\end{figure}








\subsubsection{\underline{Bra}in \underline{T}umor \underline{S}egmentation Challenge Data}

The Brain tumor segmentation
Associated with the 2012 International Conference on Medical Image Computing and Computer Assisted
Intervention (MICCAI),%
\footnote{
http://www.miccai2012.org
}
the 


\subsubsection{MS lesion segmentation challenge Challenge Data}

\section{Results}

\begin{table*}
\caption{Dice scores from the MICCAI 2012 BRATs Study}
\begin{center}
\begin{tabular*}{0.975\textwidth}{@{\extracolsep{\fill} } c c c c c c c c c}
\toprule
{} & \multicolumn{2}{c}{High-grade (real)} & \multicolumn{2}{c}{Low-grade (real)} & \multicolumn{2}{c}{High-grade (simulated)} & \multicolumn{2}{c}{Low-grade (simulated)}\\
{\bf Method} & Edema & Tumor & Edema & Tumor & Edema & Tumor & Edema & Tumor\\
\midrule
\cite{zikic2012} & {$0.70 \pm 0.09$} & {$0.71 \pm 0.24$} & {$0.44 \pm 0.18$} & {$0.62 \pm 0.27$} & {$0.65 \pm 0.27$} & {$0.90 \pm 0.05$} & {$0.55 \pm 0.23$} & {$0.71 \pm 0.20$} \\
\cite{bauer2012} & {$0.61 \pm 0.15$} & {$0.62 \pm 0.27$} & {$0.35 \pm 0.18$} & {$0.49 \pm 0.26$} & {$0.68 \pm 0.26$} & {$0.90 \pm 0.06$} & {$0.57 \pm 0.24$} & {$0.74 \pm 0.10$} \\
ANTsR & {$0.65 \pm 0.15$} & {$0.66 \pm 0.28$} & {$0.49 \pm 0.16$} & {$0.65 \pm 0.21$} & {$0.68 \pm 0.25$} & {$0.91 \pm 0.08$} & {$0.61 \pm 0.25$} & {$0.84 \pm 0.09$} \\
w/ Atropos & {$0.68 \pm 0.15$} & {$0.67 \pm 0.30$} & {$0.50 \pm 0.15$} & {$0.67 \pm 0.23$} & {$0.74 \pm 0.26$} & {$0.92 \pm 0.09$} & {$0.65 \pm 0.26$} & {$0.84\pm 0.08$} \\
\bottomrule
\end{tabular*}
\end{center}
\end{table*}

\section{Discussion and Conclusions} 

%% References with bibTeX database:

\section*{Acknowledgments}

\section*{References}

\bibliographystyle{elsarticle-harv}
\bibliography{references}


%% Authors are advised to submit their bibtex database files. They are
%% requested to list a bibtex style file in the manuscript if they do
%% not want to use model1-num-names.bst.

%% References without bibTeX database:

% \begin{thebibliography}{00}

%% \bibitem must have the following form:
%%   \bibitem{key}...
%%

% \bibitem{}

% \end{thebibliography}


\end{document}

%%
%% End of file `elsarticle-template-1-num.tex'.
