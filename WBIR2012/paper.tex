% This is LLNCS.DEM the demonstration file of
% the LaTeX macro package from Springer-Verlag
% for Lecture Notes in Computer Science,
% version 2.4 for LaTeX2e as of 16. April 2010
%
\documentclass{llncs}
%
\usepackage{makeidx}  % allows for indexgeneration
\usepackage{amssymb}
\usepackage{amsmath}
\usepackage{algorithmic,eqparbox,array}
\usepackage{algorithm}


%\floatname{algorithm}{Procedure}
\renewcommand{\algorithmicrequire}{\textbf{Input:}}
\renewcommand{\algorithmicensure}{\textbf{Output:}}
\renewcommand{\algorithmiccomment}[1]{\hfill\eqparbox{COMMENT}{// #1}}
%
\begin{document}
%
\frontmatter          % for the preliminaries
%

\mainmatter              % start of the contributions
%
\title{Diffeomorphic Directly Manipulated Free-Form Deformation Image Registration via Vector Field Flows}
%
\titlerunning{Diffeomorphic DMFFD Image Registration}  % abbreviated title (for running head)
%                                     also used for the TOC unless
%                                     \toctitle is used
%
\author{Nicholas J. Tustison\inst{1} \and Brian B. Avants\inst{2}}
%
\authorrunning{N. Tustison et al.} % abbreviated author list (for running head)
%

\institute{University of Virginia, Charlottesville VA 22903, USA,\\
\email{ntustison@virginia.edu}
\and
University of Pennsylvania,
Philadelphia PA  18104,USA,\\
\email{avants@picsl.upenn.edu}}

\maketitle              % typeset the title of the contribution

\begin{abstract}
Motivated by previous work \cite{tustison2009} and 
recent diffeomorphic image registration developments in which the 
characteristic velocity field is represented by spatiotemporal B-splines 
\cite{de-craene2011}, we present a diffeomorphic B-spline-based
image registration algorithm combining and extending these techniques.  
The advancements of the proposed framework
over previous work include a preconditioned gradient descent algorithm, 
potential spatial weighting 
of the gradient permitting (among other things) enforcement of 
stationary boundary conditions, multiresolution in terms of the B-spline
mesh, facilitating the incorporation of landmark information, 
and the possibility of modeling temporal periodicity.  In addition to
theoretical and practical discussions of our contribution, we also describe 
its parallelized implementation as open source in the Insight Toolkit.  
\keywords{B-splines, DMFFD, diffeomorphism, ITK}
\end{abstract}

\section{Introduction}
Significant algorithmic developments characterizing modern intensity-based
image registration research include the B-spline parameterized approach
(so called {\em free-form deformation}) with early contributions 
including \cite{szeliski1997,thevenaz1998,rueckert1999}.  Amongst the 
variant extensions, the 
{\em directly manipulated free-form deformation} approach  
 addressed the hemstitching issue associated with the
problematic energy topographies caused by the distribution of the uniform 
B-spline shape functions over the transformation domain \cite{tustison2009}.

Other important image registration research
reflected increased emphasis on topological transformation considerations
in modeling biological/physical systems where topology is 
consistent throughout the course of deformation or a 
homeomorphic relationship is assumed between image domains.
Methods such as LDDMM \cite{beg2005} optimize time-varying velocity field 
flows to yield diffeomorphic transformations.  Alternatively, the FFD 
variant reported in \cite{rueckert2006} enforced diffeomorphic transforms
by concatenating multiple FFD transforms, each of which is constrained
to describe a one-to-one mapping.  Another FFD registration
incorporated the recent log-Euclidean framework for enforcing diffeomorphic
transformations \cite{Modat2011}.
Recently, the work of 
\cite{de-craene2011} combined these registration concepts into a single
framework called {\em temporal free-form deformation} in which the 
time-varying velocity field characteristic of LDDMM-style algorithms
is modeled using a 4-D B-spline object (3-D + time).  Integration of 
the velocity field yields the mapping between parameterized time points.

In this work, we describe our extension to these methods.  Similar to 
\cite{de-craene2011}, we also use an $N$-D + time B-spline object to 
represent the characteristic velocity field.  However, we use the 
directly manipulated free-form deformation optimization formulation to improve 
convergence during the course of optimization.  This also facilitates
modeling temporal periodicity and
the enforcement of stationary boundaries consistent with diffeomorphic
transforms.
We also incorporate B-spline mesh multi-resolution capabilities
for increased control during registration progress.  
Most importantly, we also describe
our parallelized algorithmic implementation as open source available through the Insight Toolkit.%
\footnote{
http://www.itk.org/
}

We first describe the methodology by laying out a mathematical description 
of the various algorithmic elements coupled with implementation details
where appropriate.  This is followed by several registration examples
to provide insight into the details of our contribution. 

\section{Methods: Formulae  and Implementation}

In this section, we explain the underlying theory focusing on
differences with previous work.  We first explain how B-spline
velocity fields can be used to produce diffeomorphisms through
analytical integration involving the B-spline basis functions. 
We then show how our previous work involving optimization in 
B-spline vector spaces \cite{tustison2009} can be used for
optimization of diffeomorphisms.  Additional insight is then
gleaned by illustrating correspondence between theory and 
implementation.

\subsection{B-spline velocity field transform}

Briefly, as with other diffeomorphic formulations based on vector flows, we
assume the diffeomorphism, $\phi$, is defined on the image domain, $\Omega$, 
with stationary boundaries such that $\phi( \partial \Omega) = \mathbf{Id}$.
$\phi$ is generated as the solution of the ordinary differential
equation 
\begin{align}
  \label{eq:ode}
  \frac{d\phi(\mathbf{x}, t)}{dt} = v(\phi(\mathbf{x}, t), t),\,\,\phi( \mathbf{x}, 0 ) = \mathbf{x}
\end{align}
where $v$ is a (potentially) time-dependent smooth field, $v : \Omega \times t 
\rightarrow \mathrm{R}^d$ parameterized by $t \in [0,1]$.  Diffeomorphic mappings
between parameterized time points $\{t_a,t_b\} \in [0,1]$ 
are obtained from  Eq. (\ref{eq:ode}) through integration of the transport
equation, viz.
\begin{align}
  \label{eq:integral}
\phi(\mathbf{x},t_b) &= \phi(\mathbf{x},t_a) + \int_{t_a}^{t_b} v(\phi(\mathbf{x}), t) dt.
\end{align}

In the case of 
$d$-dimensional registration, we can represent the time-dependent velocity field  
as a $(d + 1)$-dimensional B-spline object
\begin{align}
v(\mathbf{x}, t) = \sum_{i_1=1}^{X_1}\ldots\sum_{i_d=1}^{X_d}\sum_{i_t=1}^T v_{i_1,\ldots,i_d,i_t} B_{i_t}(t) \prod_{j=1}^d B_{i_j}(x_j)
\end{align}
where $v_{i_1,\ldots,i_d,i_t}$ is a $(d+1)$-dimensional control point lattice
characterizing the velocity field and $B(\cdot)$ are the univariate B-spline
basis functions separately modulating the solution in each parametric dimension.

Although various methods exist for solving Eqns. (\ref{eq:ode}) and (\ref{eq:integral}),
we use $4^{th}$-order Runge-Kutta, i.e.
\begin{align}
  \phi_{n+1} &= \phi_{n} + \frac{1}{6}\left( k_1 + 2k_2 + 2k_3 + k_4 \right) \\
  t_{n+1} &= t_{n} + \Delta t \\ 
  \phi_0 &= \phi(t_0) 
\end{align}
where
\begin{align}
  k_1 &= v\left( \phi_{n}, t_{n} \right)\Delta t \\
  k_2 &= v\left( \phi_{n} + \frac{k_1}{2}, t_{n} + \frac{\Delta t}{2} \right)\Delta t \\
  k_3 &= v\left( \phi_{n} + \frac{k_2}{2}, t_{n} + \frac{\Delta t}{2} \right)\Delta t \\
  k_4 &= v\left( \phi_{n} + k_3, t_{n} + \Delta t \right)\Delta t
\end{align}  
which provides a less computationally expensive, more stable alternative than other numerical
methods.


%Combining Eqns. (\ref{eq:ode}) and (\ref{eq:integral}) we obtain
%\begin{align}
%\phi(\mathbf{x},t_b) = \phi(\mathbf{x},t_a) + \int_{t_a}^{t_b} \left( \sum_{i_1=1}^{X_1}\ldots\sum_{i_d=1}^{X_d}\sum_{i_t=1}^T 
%    v_{i_1,\ldots,i_d,i_t} B_{i_t}(t)\prod_{j=1}^d B_{j}(x_j) \right) dt 
%\end{align}
%Moving the integral inside the summations
%\begin{align}
%\label{eq:velocityfield}
%\phi(\mathbf{x},t_b) = \phi(\mathbf{x},t_a) + \sum_{i_1=1}^{X_1}\ldots\sum_{i_d=1}^{X_d} 
%        \underbrace{\left( \sum_{i_t=1}^T v_{i_1,\ldots,i_d,i_t} \int_{t_a}^{t_b} B_{i_t}(t) dt \right)}_{\phi'_{i_1,\ldots,i_d}} \prod_{j=1}^d B_{i_j}(x_j) 
%\end{align}
%demonstrates the sole dependency on the B-spline basis functions in
%the temporal parametric dimension.  The integral is easily calculated
%as the B-spline basis functions are piecewise polynomials.
%Also, it is seen that the 
%quantity inside the parentheses comprises the $d$-dimensional 
%control point lattice defining the diffeomorphic mapping
%defined between time points $[t_a,t_b]$ obtained by 
%integration.  Furthermore, supposing the initial diffeomorphism,
%$\phi(\mathbf{x}, t_a)$, is similarly parameterized, i.e.
%\begin{align}
%  \phi(\mathbf{x}, t_a) = \sum_{i_1=1}^{X_1}\ldots\sum_{i_d=1}^{X_d} \phi^{t_a}_{i_1,\ldots,i_d}\prod_{j=1}^d B_{i_j}(x_j), 
%\end{align}
%the diffeomorphism at $t_b$ is obtained by simple addition of the two control point 
%lattices,
%\begin{align}
%\phi(\mathbf{x},t_b) = \sum_{i_1=1}^{X_1}\ldots\sum_{i_d=1}^{X_d} \left(\phi^{t_a}+
%\phi' \right)_{i_1,\ldots,i_d} \prod_{j=1}^d B_{i_j}(x_j).
%\end{align}

\subsection{Directly manipulated free-form deformation optimization of the B-spline velocity field}

In \cite{tustison2009} 
it was observed that optimization of FFD registration with gradient descent is 
intrinsically susceptible to problematic energy topographies.  However, a
well-understood preconditioned gradient was proposed based on the work 
described in \cite{tustison2006} which substantially 
improves performance which we refer to as DMFFD image registration.  
Similarly, we propose the following velocity field control
point lattice preconditioned gradient,
$\delta v_{i_1,\ldots,i_d,i_t}$, given the similarity metric, $\Pi$,%
\footnote
{
Current options include neighborhood cross correlation (CC), mutual information (MI), and
Demons-style sum of squared differences (Demons).
}
\begin{align}
\label{eq:dmffd}
  \delta v_{i_1,\ldots,i_d,i_t} &= \left( \sum_{c=1}^{N_{\Omega} \times N_t} \left( \frac{\partial \Pi}{\partial \mathbf{x}} \right)_c B_{i_t}(t^c)\prod_{j=1}^d B_{i_j}(x_j^c)  \right. \nonumber \\
  &\cdot \left. \frac{B_{i_t}^2(t^c) \prod_{j=1}^d B_{i_j}^2 (x_j^c)} 
  {\sum_{k_1=1}^{r+1}\ldots\sum_{k_d=1}^{r+1} \sum_{k_t=1}^{r+1} B_{k_t}^2(t^c) 
  \prod_{j=1}^d B_{k_j}^2 (x_j^c)} \right) \nonumber \\
  &\cdot\left({\sum_{c=1}^{N_{\Omega}\times N_t}B_{i_t}^2(t^c) \prod_{j=1}^d B_{i_j}^2 (x_j^c)} \right) ^{-1}
\end{align}
which is a slight modification of Eqn. (21)
in \cite{tustison2009} which takes into account the temporal locations of the 
dense gradient field sampled in $t \in [0,1]$. $N_t$ and $N_\Omega$ are the number
of time point samples and the number of voxels in the reference image domain, respectively.
$r$ is the spline order in all dimensions%
\footnote{
Spline orders can be specified separately for each dimension but, for simplicity,
we only specify a single spline order.  However, the framework can easily 
accommodate different spline orders.
}
and $c$ indexes the spatio-temporal dense metric
gradient sample.
Combining DMFFD optimization with the B-spline velocity field transform discussed in
the previous subsection, we outline the iterative routine for optimizing the 
velocity field control in Algorithm \ref{alg1}.

\begin{algorithm}                    
\caption{Gradient descent optimization of the B-spline velocity field}        
\label{alg1}      
\begin{algorithmic}[1]
\REQUIRE images $\mathcal{I}$ and $\mathcal{J}$
\REQUIRE  user-specified similarity metric $\Pi$ 
\REQUIRE  velocity field mesh size and multiresolution schedule, $R$
\REQUIRE  number of time point samples, $N_t$ (default: 4)
\REQUIRE  spline order (default: 3)
\REQUIRE  gradient step size, $\lambda$ (default: 0.25)
\ENSURE velocity field control point lattice, $v$
\STATE $v \gets \mathbf{0}$ \COMMENT{Initialize velocity field}
\FOR{number of resolution levels} 
\FOR{number of iterations for current level} 
  \STATE $G \gets []$  \COMMENT{Initialize dense gradient storage array}
\FOR{$t = 1 \to N_t$ } 
  \STATE $t' \gets \frac{t-1}{N_t-1}$  \COMMENT{Scale time point to $[0,1]$}
  \STATE $\mathcal{I}' \gets \mathcal{I} \circ \phi(\mathbf{x}, t')$  \COMMENT{Warp $\mathcal{I}$ to the current time point}
  \STATE $\mathcal{J}' \gets \mathcal{J} \circ \phi^{-1}(\mathbf{x}, 1-t')$ \COMMENT{Warp $\mathcal{J}$ to the current time point}
  \STATE $G[t] \gets \frac{\partial \Pi(\mathcal{I}',\mathcal{J}')}{\partial \mathbf{x}}$  \COMMENT{Store dense similarity metric gradient}
\ENDFOR  
  \STATE $\delta v \gets B(G)$ \COMMENT{Calculate velocity lattice gradient (Eqn. (\ref{eq:dmffd}))}
  \STATE $v \gets v + \lambda\delta v$ \COMMENT{Take a step in the gradient direction}
\ENDFOR  
  \STATE $v \gets R(v)$         \COMMENT{Refine lattice based on schedule}
%  \IF {converged}
%    \RETURN $v$
%  \ENDIF
\ENDFOR  
\end{algorithmic}
\end{algorithm}

\subsection{Implementation}

As mentioned previously, the registration algorithm has been implemented in the
Insight Toolkit and consists of the following major classes: 
\begin{itemize}
  \item \verb#itk::TimeVaryingVelocityFieldIntegrationImageFilter#
  \item \verb#itk::TimeVaryingBSplineVelocityFieldTransform#
  \item \verb#itk::TimeVaryingBSplineVelocityFieldImageRegistrationMethod#
  \item \verb#itk::TimeVaryingBSplineVelocityFieldTransformParametersAdaptor#
  \item \verb#itk::BSplineScatteredDataPointSetToImageFilter#
\end{itemize}
The \verb#TimeVaryingVelocityFieldIntegrationImageFilter# class implements
the Runge-Kutta integration described earlier.  Given a 
velocity field control point lattice as 
input and the lower and upper integration limits, integration is performed 
in a multi-threaded fashion (since each point in the domain
can be integrated separately).  The \verb#TimeVaryingBSplineVelocityFieldTransform# 
is derived from the base transform class which handles the mapping of geometric primitives
for warping images.  The DMFFD gradient calculation  is handled by the class \verb#BSplineScatteredDataPointSetToImageFilter# (cf. Eqn. \ref{eq:dmffd} and line 11
of Algorithm \ref{alg1}) which takes as input the dense similarity metric and an optional
weighting for each gradient sample.  This permits enforcement of stationary physical
boundaries by specifying a zero gradient on the boundaries and a large weighting.
Coordinating all the elements of image registration is the
\verb#TimeVaryingBSplineVelocityFieldImageRegistrationMethod# class which
encapsulates Algorithm \ref{alg1}.  The resolution scheduling is determined
by the \verb#TimeVaryingBSplineVelocityFieldTransformParametersAdaptor# class.
We provide access to the new ITK 
registration framework (including the B-spline velocity field transform) 
through the command line module \verb#antsRegistration# available both in ANTs%
\footnote{
http://www.picsl.upenn.edu/ANTs
}
and accompanied by a technical report offered through the Insight Journal
\cite{tustison2012}.%
\footnote{
http://www.insight-journal.org
}
The interested reader should consult \cite{tustison2012} for further details on 
actual usage.

\section{Registration Examples}

\section{Discussion and Conclusions}
This work constitutes an advantageous combination of the continuous aspects of
B-splines with the diffeomorphic registration framework via vector field 
flows.  We incorporate
DMFFD optimization of the B-spline velocity field which facilitates 
convergence and permits enforcement of stationary boundary conditions.  
While not discussed, further advantages include incorporation of temporal 
periodicity in dealing with the possibility of multiple images describing
periodic motion (e.g. cardiac or pulmonary motion).  Although not discussed
in this work, a future publication will explore these possibilities in 
greater detail. 

\bibliographystyle{splncs03}
\bibliography{references}

\end{document}
