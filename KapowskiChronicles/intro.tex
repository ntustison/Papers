\section{Introduction}

Neuroscientific investigations into cortical morphological changes/differences have illuminated
interesting correlations with normal and pathological neurodevelopment in 
addition to cognitive function.
Historically rooted in the meticulous work of von Economo \citep{economo2008},
computational methods for modeling the
cortical gray matter have resulted in numerous studies 
exploring such relationships.  
These analysis techniques
have been instrumental in demonstrating 
the presence of regional cortical abnormalities in such conditions as 
Huntington's disease \citep{rosas2002,rosas2005,selemon2004}, 
schizophrenia \citep{nesvag2008}, bipolar disorder \cite{lyoo2006}, Alzheimer's disease and frontotemporal
dementia \citep{du2007,dickerson2009}, Parkinson's disease \citep{jubault2011},
multiple sclerosis \citep{ramasamy2009}, autism \citep{chung2005,hardan2006},
migraines \citep{dasilva2007}, chronic smoking \citep{kuhn2010}, alcoholism \citep{fortier2011},
cocaine addiction \citep{makris2008}, Tourette syndrome in children \citep{sowell2008},
scoliosis in
female adolescents \citep{wang2012}, obsessive compulsive
disorder \citep{shin2007}, ADHD \citep{almeida-montes2012}, obesity \citep{raji2010}, and heritable \citep{peterson2009}
and elderly \citep{ballmaier2004} depression.  Further study 
topics exploring correlations with cortical thickness include
normal aging \citep{kochunov2011},
gender differences \citep{luders2006a},
anatomical asymmetries 
\citep{luders2006,amunts2007}, intelligence \citep{shaw2006}, athletic
ability \citep{wei2011}, musical ability \citep{bermudez2009,foster2010}, 
white collar criminals \citep{raine2011}, and Tetris-playing
ability in female adolescents \citep{haier2009}.  Additionally,
recent studies have demonstrated possible functional 
connectivity relationships using cortical thickness measures
\cite{worsley2005,lerch2006,he2007}.
Although these study findings
are subject to debate and interpretation \citep{gernsbacher2007}, 
the availability of quantitative
computational methods for extracting such information
has proven invaluable for developing and refining fundamental 
neuroscience hypotheses.

Broadly characterized as surface mesh-based or volumetric \citep{scott2009,clarkson2011}, such 
techniques have been introduced and extensively developed over 
the last $10+$ years.  Representative of the former is the
Freesurfer%
\footnote{
http://surfer.nmr.mgh.harvard.edu/
}
cortical modeling software package \citep{dale1999,fischl1999,fischl2000,fischl2002,fischl2004}
which owes its popularity to public availability, excellent documentation, 
good performance, and  integration with other toolkits, such as the extensive FMRIB software 
library (FSL) \citep{smith2004}.  Similar to other surface
approaches (e.g. \cite{davatzikos1996,magnotta1999,macdonald2000,kim2005}), the pial
and white matter surfaces from individual subject MR data are modeled with polygonal meshes  
which are then used to determine local cortical thickness based on a specified correspondence between 
the surface models.

Image volumetric (or meshless) techniques are varied both in algorithmic terms as well as
the underlying definition of cortical thickness.  An early, foundational technique is the 
method of \cite{jones2000} in which the inner and outer surface geometry is used to determine the
solution to Laplace's equation in which thickness is measured by integrating along the 
tangents of the resulting field lines spanning the boundary surfaces.  Subsequent contributions
improved upon the original formulation.  For example, in \cite{yezzi2003}, an Eulerian PDE approach
was proposed to facilitate the computation of correspondence paths.  Extending the surface-based
work of \cite{macdonald2000}, the hybrid approach of
\cite{kim2005} uses the discrete Laplacian field to deform the white matter surface mesh towards the 
pial surface.    Although the Laplacian-based approach has several advantages
including generally lower computational times and 
non-crossing correspondence paths, direct correlative assessments with histology
are difficult as the quantified distances 
are not necessarily Euclidean.  Other volumetric algorithms have employed coupled
level sets \citep{zeng1999}, model-free intelligent search strategies either normal to 
the gray-white matter interface \citep{scott2009} or using a min-max rule \citep{clement-vachet2011}.
Most relevant to this work is the DiReCT (Diffeomorphic Registration-based 
Cortical Thickness) algorithm proposed by \cite{das2009} in which a 
a registration-based approach proposed where the derived diffeomorphic mapping between the 
white and pial matter surfaces is used to propagate thickness values 
through the cortical gray matter.

Despite the numerous techniques that have been proposed in the literature (of which
only a fraction has been cited), several requisite processing components are 
common to many of them which are prior to quantification of cortical thickness.
These include inhomogeneity correction, skull stripping, and $n$-tissue segmentation 
for differentiating the gray and white matters.  For post-processing statistical analysis 
across populations, construction of population-specific unbiased templates
is also  beneficial.
In addition, some of these steps might include a crucial registration component, e.g. 
propagating template-based tissue priors for improved segmentation.  The requisite
additional components coupled with the general lack of availability of published
algorithms \cite{kovacevic2006} inhibits performing studies by external researchers 
and makes comparative evaluations difficult.  For example, one recent evaluation 
study \citep{clarkson2011} compared
Freesurfer (a surface-based method) with two volumetric methods \citep{jones2000,das2009}.
Whereas the entire Freesurfer processing pipeline has been made publicly available, 
tuned by the original authors in terms of implementation, and described in great detail 
(specifically in terms of suggested parameters); both volumetric methods were 
implemented solely by the authors of the evaluation (not the actual software developers) using 
unspecified parameters making the comparisons less than ideal.

In this work we describe our cortical thickness software pipeline which 
we have made open source as part of our Advanced Normalization Tools (ANTs) 
repository.  This includes all the necessary preprocessing steps consisting
of well-vetted previously published algorithms for bias correction \cite{tustison2010},
brain extraction \cite{avants2010a}, $n$-tissue segmentation \cite{avants2011a},
template construction \cite{avants2010}, and image normalization \cite{avants2011}.
We also describe improvements made to the original DiReCT algorithm \cite{das2009}.
Equally as important, we demonstrate how to coordinate
these pipeline components and provide a set of useful parameters
which are used to analyze the publicly available IXI data set. This is 
encapsulated in a well-documented shell script which is also available in ANTS. 
Furthermore, we provide all the derived image data and other processing scripts on 
a github repository specifically meant for this publication.  This permits
the set of results described in this work to be fully reproducible thus
permitting other researchers to use a complete volumetric pipeline for measuring
cortical thickness.








%This might explain why there have been few evaluation studies.  For example, in
%comparing volumetric and surface-based methods, \cite{clarkson2011} use the
%Freesurfer implementation but rely on their own implementations of published 
%literature which might not be an unbiased evaluation particularly given the 
%complexity of the underlying registration algorithmic work in \cite{das2009}.
%Based solely on the number of citations in the literature,
%the Freesurfer%
%\footnote{
%http://surfer.nmr.mgh.harvard.edu/
%} 
%cortical modeling software package is perhaps the most ubiquitous 
%with several publications detailing various stages of development and
%methodology \citep{dale1999,fischl1999,fischl2000,fischl2002,fischl2004}.  
%Public availability, excellent documentation, good performance, and 
%integration with other toolkits, such as the extensive FMRIB software 
%library (FSL) \cite{smith2004},
%have contributed to its popularity.  



%Freesurfer's individual brain processing pipeline begins with segmentation
%and surface modeling of the gray white matter interface.  Gradients 


%\begin{table*}
%\caption{Component-wise breakdown of reported cortical thickness methods reported in the literature.}
%\begin{tabular*}{0.95\textwidth}{@{\extracolsep{\fill}} c c c c c p{4.5cm} }
%\hline
%{\bf Algorithm} & {\bf V/S} & {\bf Bias correction} & {\bf Brain extraction} & {\bf Segmentation} & \multicolumn{1}{c}{\bf Additional notes} \\
%\hline
%BRAINSURF \cite{magnotta1999} & S & {} & none & Discriminate analysis \cite{harris1999} & {} \\
%ASP \cite{macdonald2000} & S & {} & N3$^\dagger$ \cite{Sled1998} & Neural-nets \cite{ozkan1993} & {}\\
%Laplace \cite{Jones2000} & V & {} & none & simple thresholding & {} \\
%CLASP \cite{kim2005} & V/S & {} & N3$^\dagger$ \cite{Sled1998} & K-NN \cite{cocosco2003} & { Partial volume classification of GM/CSF \cite{Choi1991} is used to facilitate reconstructing the pial surface. }\\
%\cite{hutton2005} & V & {} & {} & {} & {} \\
%DiReCT \cite{das2009} & V & {} & none & FAST$^\dagger$ \cite{zhang2001} & {} \\
%\cite{scott2009} & V & {} & none & E-M Bayes \cite{pokric2001}  & {} \\
%\cite{acosta2009} & V & {} & \cite{van-leemput1999a} & E-M Bayes. \cite{van-leemput1999} & {}\\
%\cite{lerch2005}$^{BrainVisa}$ & {} & {} & {} & {} & {} \\
%CRUISE\cite{han2004,tosun2006} & S& {}  & TOADS$^\dagger$ \cite{bazin2007} & {} \\
%
%CLADA\cite{nakamura2011} & S& {}  & PABIC$^\dagger$ \cite{styner2000} & \cite{nakamura2009} & {longitudinal analysis}\\
%%SIENA$^\dagger$\cite{smith2002} & {} & {} & FAST$^\dagger$ \cite{zhang2001} & {longitudinal analysis}\\
%\hline
%\end{tabular*}
%\end{table*}
%
%Open source packages:
%\begin{itemize}
%\item (http://www.ncbi.nlm.nih.gov/pubmed/15957597) TINA - be sure to read the reviews which aren't very good.
%\item (http://www.nitrc.org/projects/arctic/ %http://www.na-mic.org/Wiki/index.php/UNC_ARCTIC_Tutorial) 
%ARCTIC (Automatic Regional Cortical ThICkness)
%\item Brain Voyager (Goebel?)
%\item TOADS-CRUISE (Tosun et al.) http://www.nitrc.org/projects/toads-cruise/
%\item GAMBIT
%\item http://www.bic.mni.mcgill.ca/thickness\_population\_simulation/ \cite{lerch2005}
%\item Be sure to email Vincent Magnotta to see if they have open source tools in Brains for estimation of cortical thickness
%\end{itemize}
