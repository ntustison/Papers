%% This is file `elsarticle-template-1-num.tex',
%%
%% Copyright 2009 Elsevier Ltd
%%
%% This file is part of the 'Elsarticle Bundle'.
%% ---------------------------------------------
%%
%% It may be distributed under the conditions of the LaTeX Project Public
%% License, either version 1.2 of this license or (at your option) any
%% later version.  The latest version of this license is in
%%    http://www.latex-project.org/lppl.txt
%% and version 1.2 or later is part of all distributions of LaTeX
%% version 1999/12/01 or later.
%%
%% The list of all files belonging to the 'Elsarticle Bundle' is
%% given in the file `manifest.txt'.
%%
%% Template article for Elsevier's document class `elsarticle'
%% with numbered style bibliographic references
%%
%% $Id: elsarticle-template-1-num.tex 149 2009-10-08 05:01:15Z rishi $
%% $URL: http://lenova.river-valley.com/svn/elsbst/trunk/elsarticle-template-1-num.tex $
%%

\documentclass[preprint,authoryear,review,12pt]{elsarticle}
%\documentclass[final,5p,times,twocolumn]{elsarticle}

%% Use the option review to obtain double line spacing
%% \documentclass[preprint,review,12pt]{elsarticle}

%% Use the options 1p,two column; 3p; 3p,twocolumn; 5p; or 5p,twocolumn
%% for a journal layout:
%% \documentclass[final,1p,times]{elsarticle}
%% \documentclass[final,1p,times,twocolumn]{elsarticle}
%% \documentclass[final,3p,times]{elsarticle}
%% \documentclass[final,3p,times,twocolumn]{elsarticle}
%% \documentclass[final,5p,times]{elsarticle}
%% \documentclass[final,5p,times,twocolumn]{elsarticle}


\usepackage{color}
\usepackage{multirow,booktabs,ctable,array}
\usepackage{lscape}
\usepackage{amsmath}
\usepackage{lineno}
\usepackage{ulem}
\usepackage{setspace}
\usepackage{listings}
\usepackage{float}


\floatstyle{plain}
\newfloat{command}{thp}{lop}
\floatname{command}{Command}

%\usepackage[nomarkers,notablist]{endfloat}

%% if you use PostScript figures in your article
%% use the graphics package for simple commands
%% \usepackage{graphics}
%% or use the graphicx package for more complicated commands
%% \usepackage{graphicx}
%% or use the epsfig package if you prefer to use the old commands
%% \usepackage{epsfig}

%% The amssymb package provides various useful mathematical symbols
\usepackage{amssymb}
%% The amsthm package provides extended theorem environments
% \usepackage{amsthm}
 
 \usepackage{makecell}

%% The lineno packages adds line numbers. Start line numbering with
%% \begin{linenumbers}, end it with \end{linenumbers}. Or switch it on
%% for the whole article with \linenumbers after \end{frontmatter}.
%% \usepackage{lineno}

%% natbib.sty is loaded by default. However, natbib options can be
%% provided with \biboptions{...} command. Following options are
%% valid:

%%   round  -  round parentheses are used (default)
%%   square -  square brackets are used   [option]
%%   curly  -  curly braces are used      {option}
%%   angle  -  angle brackets are used    <option>
%%   semicolon  -  multiple citations separated by semi-colon
%%   colon  - same as semicolon, an earlier confusion
%%   comma  -  separated by comma
%%   numbers-  selects numerical citations
%%   super  -  numerical citations as superscripts
%%   sort   -  sorts multiple citations according to order in ref. list
%%   sort&compress   -  like sort, but also compresses numerical citations
%%   compress - compresses without sorting
%%
%% \biboptions{comma,round}

% \biboptions{}

\providecommand{\OO}[1]{\operatorname{O}\bigl(#1\bigr)}

\graphicspath{{./Figures/}
                          }

\long\def\symbolfootnote[#1]#2{\begingroup%
\def\thefootnote{\fnsymbol{footnote}}\footnote[#1]{#2}\endgroup}

    \usepackage{color}

    \definecolor{listcomment}{rgb}{0.0,0.5,0.0}
    \definecolor{listkeyword}{rgb}{0.0,0.0,0.5}
    \definecolor{listnumbers}{gray}{0.65}
    \definecolor{listlightgray}{gray}{0.955}
    \definecolor{listwhite}{gray}{1.0}

\newcommand{\lstsetcpp}
{
\lstset{frame = tb,
        framerule = 0.25pt,
        float,
        fontadjust,
        backgroundcolor={\color{listlightgray}},
        basicstyle = {\ttfamily\scriptsize},
        keywordstyle = {\ttfamily\color{listkeyword}\textbf},
        identifierstyle = {\ttfamily},
        commentstyle = {\ttfamily\color{listcomment}\textit},
        stringstyle = {\ttfamily},
        showstringspaces = false,
        showtabs = false,
        numbers = none,
        numbersep = 6pt,
        numberstyle={\ttfamily\color{listnumbers}},
        tabsize = 2,
        language=[ANSI]C++,
        floatplacement=!h,
        caption={},
        captionpos=b,
        }
}


\journal{Neuroimage}

\begin{document}


\begin{frontmatter}

\title{Neuroanatomical priors improve image-based cortical thickness computation: ROI-DiReCT}



%\author[label1]{Nicholas J.~Tustison\fnref{label0}}
%%  \ead{ntustison@virginia.edu}
%  \fntext[label0]{\scriptsize Corresponding author:  PO Box 801339, Charlottesville, VA 22908; T:  434-924-7730; email address:  ntustison@virginia.edu }
%\author[label2]{Brian B.~Avants}
%\author[label2]{Philip A.~Cook}
%\author[label3]{Junghoon Kim}
%\author[label3]{John Whyte}
%\author[label2]{James C.~Gee}
%\author[label1]{James R.~Stone}
%
%\address[label1]{Department of Radiology and Medical Imaging, University of Virginia, Charlottesville, VA}
%\address[label2]{Penn Image Computing and Science Laboratory, University of Pennsylvania,
%                Philadelphia, PA}
%\address[label3]{Moss Rehabilitation Research Institute, Albert Einstein Healthcare Network, Philadelphia, PA}



%\maketitle

%\linenumbers


\begin{abstract} 
Cortical thickness measures derived from structural MRI demonstrate
reproducible associations with IQ, aging and disease and may be a
valuable biomarker for intervention studies.  Thickness of the
cortical mantle also varies predictably with neuroanatomical regions
as shown by von Economo in 1927.  Recent research demonstrates that
training segmentation data is able to automatically partition the
cortex into major neuroanatomical divisions.  The correlation of
neuroanatomical region and cortical thickness implies that cortical
labeling may improve cortical thickness quantification by introducing
additional prior information.  We thus propose a new solution,
ROI-DiReCT, to image-based cortical thickness estimation (ICTE)
that leverages recent advances in multi-template labeling, distributed
computing and reproducible research to both extend and validate prior
work.  We use an open-source image processing toolkit and public data
to test the hypothesis that additional priors introduced through
cortical parcellation can be used in a divide-and-conquer strategy to
improve the accuracy and speed of ICTE.  For instance, the expected
range of cortical thickness in post-central gyrus differs from that of
inferior temporal gyrus and, via ROI-DiReCT, this information may
be included in the thickness estimation.  We use the semi-manually labeled NIREP
data to contrast whole-brain ICTE with those gained by ROI-DiReCT
with respect to known patterns of cortical thickness values as
provided by von Economo and others.  We also use multi-template
labeling, based on NIREP, to automatically label the T1 images from
the multi-modality reproducibility dataset.  This allows us to compare
the reproducibility of our whole-brain and ROI-DiReCT ICTE.
ROI-DiReCT provides more stable and neuroanatomically plausible
results.
\end{abstract}

\begin{keyword}
%% keywords here, in the form: keyword \sep keyword
\end{keyword}

\end{frontmatter}
%
%
\newpage


%% MSC codes here, in the form: \MSC code \sep code
%% or \MSC[2008] code \sep code (2000 is the default)

%%
%% Start line numbering here if you want
%%
% \linenumbers

%% main text

\section{Introduction}
%State the objectives of the work and provide an adequate background, avoiding a detailed literature survey or a summary of the results.

in pumed, look at cortical thickness parcellation

\section{Material and Methods}


\section{Discussion} 
%This should explore the significance of the results of the work, not repeat them. A combined Results and Discussion section is often appropriate. Avoid extensive citations and discussion of published literature.

\section{Conclusions}

%% The Appendices part is started with the command \appendix;
%% appendix sections are then done as normal sections
%% \appendix

%% \section{}
%% \label{}

%% References
%%
%% Following citation commands can be used in the body text:
%% Usage of \cite is as follows:
%%   \citep{key}          ==>>  [#]
%%   \cite[chap. 2]{key} ==>>  [#, chap. 2]
%%   \citet{key}         ==>>  Author [#]

%% References with bibTeX database:

\section*{Acknowledgments}
All visualizations were performed using ITK-SNAP%
\footnote{
http://www.itksnap.org/
}
\citep{Yushkevich2006} and
DTI-TK.%
\footnote{
http://www.nitrc.org/projects/dtitk/
}
We also gratefully acknowledge Dr. Niels van Strien of the Norwegian University of Science and Technology
who assisted in packaging the template construction algorithm in the very useful script \verb#buildtemplateparallel.sh#
which is publicly available in ANTs.

\section*{References}

\bibliographystyle{elsarticle-harv}
\bibliography{references}


%% Authors are advised to submit their bibtex database files. They are
%% requested to list a bibtex style file in the manuscript if they do
%% not want to use model1-num-names.bst.

%% References without bibTeX database:

% \begin{thebibliography}{00}

%% \bibitem must have the following form:
%%   \bibitem{key}...
%%

% \bibitem{}

% \end{thebibliography}


\end{document}

%%
%% End of file `elsarticle-template-1-num.tex'.
